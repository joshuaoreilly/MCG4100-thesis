\section{Introduction} \label{sec:introduction}

Legged robots have made leaps and bounds in progress over the past decade, from the explosive and agile MIT Cheetah 3 to the deliberate and robust ANYmal \cite{bledt_mit_2018}\cite{hutter_anymal_2016}.  The majority of available legged robots are quadrupeds, robots composed of a chassis and four articulated legs. Many are used solely in a research environment, such as MIT Cheetah 3, HyQ2Max, Oncilla, Mini Cheetah, Stanford Doggo, and Solo \cite{bledt_mit_2018,semini_design_2017,sprowitz_oncilla_2018,katz_mini_2019,kau_stanford_2019,grimminger_open_2020}. ANYmal and Boston Dynamic's Spot have found footholds in the market for plant inspection \cite{hutter_anymal_2016}\cite{noauthor_boston_nodate}. While established applications lack diversity, design objectives do not; MIT Cheetah 3 is dynamic and mobile, while ANYmal and HyQ2Max are designed for maximum robustness \cite{bledt_mit_2018}\cite{semini_design_2017}\cite{hutter_anymal_2016}. Mini Cheetah, Stanford Doggo and Solo keep manufacturing and maintenance costs low to improve accessibility \cite{katz_mini_2019}\cite{kau_stanford_2019}\cite{grimminger_open_2020}. GOAT prioritises omnidirectional mobility and force sensitivity to maximize controllability and animal-like agility \cite{kalouche_design_2016}.

While some researchers have tackled developing a general framework for robot design and evaluation, legged robots tend to exist in their own bubbles with respect to the evaluation of their design \cite{semini_design_2017}. The force-to-body-weight ratio as defined for MIT Cheetah 3 is functionally identical to the limb acceleration used by GOAT \cite{bledt_mit_2018}\cite{kalouche_design_2016}. Cost of Transportation is perhaps the most commonly used metric and evaluates energy efficiency, a crucial weakness of legged robots when compared to wheeled ones, yet it is adopted by less than half the analyzed robots.

This thesis will aggregate the various performance metrics used by legged robots. It will then use a subset of these metrics to select the ideal leg topology for a quadruped performing litter collection in a beachfront environment.

\begin{comment}
Legged robots have made leaps and bounds in progress over the past decade. While they demonstrate impressive dynamic capabilities, legged robots are not commonly used in industrial applications for a number of reasons, including cost, high power consumption, and mechanical complexity and reduced robustness when compared to wheeled or tracked alternatives. The majority of available legged robots are quadrupeds, robots composed of a chassis and four articulated legs. Most are used solely in a research environment, such as include MIT Cheetah 3, HyQ2Max, Oncilla, Mini Cheetah, Stanford Doggo, and Solo \cite{bledt_mit_2018,semini_design_2017,sprowitz_oncilla_2018,katz_mini_2019,kau_stanford_2019,grimminger_open_2020}. ANYmal and Boston Dynamic's Spot have found footholds in the market for plant inspection \cite{hutter_anymal_2016}\cite{noauthor_boston_nodate}.

A major cost for legged robots are their actuators. High torque density electric motors are usually employed to allow for dynamic movements. These actuators can represent upwards of 70\% of the total cost in some quadrupeds \cite{katz_mini_2019}. While most robots employ three actuators per leg, some reduce this number to two in order to bring the cost of parts down. This generally results in less degrees of freedom and limited mobility. An alternative approach is to use less expensive hobby servo motors, however these lack the necessary torque and precision to allow for robust and controlled motion \cite{sprowitz_oncilla_2018}.

The topology developed in this thesis will attempt to tackle cost as a barrier-to-entry for legged robots. It will do so by exploring how a minimal number of actuators can be arranged to maintain sufficient mobility and performance while driving the overall cost down. As multiple topologies will be compared, this work focuses moreso on breadth than depth, and thus a simple working environment will be employed; the topologies will be evaluated for a beach cleanup application, where stability and energy efficiency take precedence over large dynamic maneuvers and ability to tackle complex terrain.
\end{comment}