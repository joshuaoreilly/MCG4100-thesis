\section{Discussion}

The Jansen linkage was first used by Theo Jansen in the Strandbeest, a mobile art piece that could traverse the environment using nothing more than wind power \cite{jansen_strandbeest_nodate}. The philosophy of efficiency is certainly confirmed in the simulation results. The Jansen linkage leg topology expends approximately a third the energy of the 2-DoF series-articulated leg topology, with $4.2W$ and $15.5W$ average power consumption respectively. This is explained both by having an smaller average actuator torque, as well as using a single actuator instead of two. Having less actuators, as well as a lower required torque, are equally beneficial when comparing the leg topologies' values of $\phi$; $\phi_{jansen}$ is under a fifth the value of $\phi_{2-DoF}$. The total cost of actuators to power the Jansen linkage would thus be significantly lower than the cost to actuate the 2-DoF series-articulated leg.

Where the Jansen mechanism leads in power consumption and actuator cost, it falls behind in workspace volume, force envelope volume and proprioceptive force sensitivity. Both workspace volume $V$ and force envelope volume $\Psi$ were not measured numerically as the result can be determined heuristically; the Jansen linkage is constrained to a predetermined foot trajectory and as such, has a vastly inferior workspace and by extension force envelope volume. It is worth noting that, while volume is a three-dimensional measure, leg movement was constrained during simulations to the sagittal plane, and the hip adduction/abduction is assumed to be controlled by other means, such as the implementations presented in Appendix \ref{app:A}; both leg's out of sagittal plane mobility are assumed to be equal. The Jansen linkage also demonstrates worse proprioceptive force sensitivity $\Pi$ than the 2-DoF series-articulated leg topology, with $0.0185Nm$ and $0.1929Nm$ respectively. This may be explained by antagonistic forces consuming some of the input force \cite{hubicki_atrias_2016}. The magnitude of the difference is dubious, however, as Kalouche found a proprioceptive force sensitivity for a 2-DoF series-articulated leg topology of $0.0667Nm$ \cite{kalouche_design_2016}. Either value remains higher than that of the Jansen linkage.

From Section \ref{sec:app-metrics}, it was determined that the robot should be able to navigate mildly unstructured but not overly hostile terrain. It should be able to turn and move at a maximum velocity that is safe for operation near humans, and operate for a reasonable period of time on battery or solar power. These design criteria suggest, in tandem, an emphasis on energy efficiency with only mild requirements in terms of mobility. Further, while proprioceptive force sensitivity is essential to permit accurate force control, external sensors can be used to measure foot contact forces, wind forces, etc. at the cost of additional design complexity. While the 2-DoF series-articulated leg topology has far superior leg mobility, measured by force envelope volume and workspace volume, the selected working environment is only mildly unstructured, and thus the additional mobility is of limited benefit. In contrast, the 2-DoF series-articulated topology has significantly higher power consumption. The Jansen linkage is therefore the preferred option for a robot performing litter collection in a beachfront environment.

\subsection{Limitations}

The modelling and simulation of both leg topologies suffered from various methodological limitations. A flagrant example is the chosen foot trajectory; while the Jansen linkage is constrained to a single foot trajectory, the 2-DoF series-articulated leg topology has a significantly larger work-space to play in, and therefore there may be a more efficient trajectory to follow than that of the Jansen linkage. For example, a foot trajectory which barely rises off the ground is likely more efficient than that of the Jansen linkage, which rises quite significantly in the $y$ direction.

The knee actuator for the 2-DoF series-articulated leg is aligned coaxially with the hip actuator in accordance with most series-articulated legged robots \cite{bledt_mit_2018}\cite{katz_mini_2019}\cite{grimminger_open_2020}. This approach requires the use of a pulley or linkage to manipulate the knee remotely. The energy which would be lost in this transfer mechanism is not considered when calculating the proprioceptive force transparency. Equally, the mass of the transfer mechanism is not considered, although this is likely of insignificant amplitude when compared to the torso and leg masses.

The use of polynomial functions to approximate the node positions for the Jansen linkage and actuator positions for the 2-DoF series-articulated leg introduced inaccuracies. Figures \ref{fig:5_2dof_angles_approx_7_15} and \ref{fig:5_2dof_torques_approx_7_15} demonstrate how the degree of the polynomial function influences both the accuracy of the initial fit, as well as the oscillatory behavior that appears when the fit is derived. Additionally, Figure \ref{fig:5_x5_jump} demonstrates how applying the polynomial approximation to a single cycle can introduce instantaneous position jumps which propagate into accelerations and finally actuator torques. These were a consequence of modelling backwards; the simulation procedure began with the desired position of each node or actuator angle at each time step and ended with the actuator torques, instead of applying actuator torques and calculating the resulting node or actuator angle. Since this approach does account for real actuator acceleration and torque limits, the rise time is effectively zero between the current state and the next, and thus joint and node accelerations explode. Properly modelling both legs using a controller with rate limiters on joint accelerations and torques would resolve these irregularities.

Within the selected simulation procedure, two major errors occured. First, the non-linear terms $C(q,\dot{q})$ in the 2-DoF series-articulate model are always zero; the equations were likely miscalculated. The torso velocity was set to $0.3\frac{m}{s}$; from this, the required velocity of the foot while on the ground was determined. Tn the simulation for the Jansen linkage, however, the angular velocity during the ground contact phase was set as constant instead of the foot velocity. While this would result in the same torso velocity as it is akin to using the average angular velocity at all times, the foot ground velocity should be not constant. If simulated correctly, the angular velocity would vary, and thus the results may have varied.

\subsection{Conclusion}

The performance metrics presented in this thesis cover the diverse goals of legged robots, from explosive and dynamic quadrupeds, to simple and cost effective ones. A subset of these metrics were selected to evaluate two leg topologies for a beachfront litter collection application; a Jansen linkage based leg and a 2-DoF series-articulated leg. The Jansen linkage was successfully determined to be the optimal leg design for this application, with some caveats in the modelling and simulation of the topologies. Future work correcting these shortcomings would further validate the choice of leg topology for the given use-case and performance metrics. Additionally, analyzing the other leg topology archetypes presented would either reinforce the selected leg topology or determine a superior one.